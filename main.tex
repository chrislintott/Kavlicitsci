1. Introduction

Citizen science in the form of distributed data analysis has made significant contributions to astronomical science through projects such as Galaxy Zoo which allow many hundreds of thousands of people to make contributions. As astronomers continue to develop surveys which will produce ever growing volumes of (often) publicly accessible data, it seems likely that further opportunities for such collaboration will occur, though the increasing sophistication of machine learning and the need in some cases for rapid classification may well requiring increasingly sophisticated frameworks to be put in place. This paper is the result of discussions at a workshop held at St Catherine's College in Oxford from April 15th-17th 2015, hosted by the Zooniverse and sponsored by the Kavli Foundation, at which participants sought to explore the possibilities for citizen science with the crop of surveys currently under development. 

2. The problem : Surveys and big data

3. Zooniverse architecture

4. Real time classification

5. Machine Learning

6. Recommendations